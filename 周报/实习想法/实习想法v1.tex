\documentclass[11pt]{ctexart}

\usepackage{multicol}
%\usepackage{mwe}
\usepackage{subfigure}
\usepackage{mathtools}
\usepackage{graphicx}
\usepackage{amsmath}
\usepackage{mathrsfs}
\usepackage[top=0.5in,bottom=1in,left=1in,right=1in]{geometry}
\usepackage{pdflscape}
\usepackage{times}
\usepackage{bm}
%\usepackage{setspace}
\usepackage{color}
\usepackage{caption}
\usepackage{amsmath}
\usepackage{amssymb}
\usepackage{CJK}
%\usepackage[final]{pdfpages}
\usepackage{listings}
\usepackage{textcomp}
\usepackage{xcolor}
\usepackage{algorithm2e}

\usepackage{algorithmicx}
\usepackage{algpseudocode}
\usepackage{hyperref}

\hypersetup{hidelinks,
	colorlinks=true,
	allcolors=black,
	pdfstartview=Fit,
	breaklinks=true}

\pagestyle{plain}




\begin{document}

\title{实习报告2022-02-21}
\author{宋欣源}
\date{\today}

\maketitle % need full-width title

\CTEXsetup[format={\Large\bfseries}]{section}

\section{第一,综述}

下面对于实习的工作做一个报告。

\section{第二,期货策略}
\subsection{综述}

从1月开始,构建基于L1和五档数据的tick级高频期权策略,取得了很大的成果,在各种策略搭建上都有了较大进展。


\subsection{第一部分,数据清洗和采集}
主要策略数据依赖于期权逐笔成交数据,部分数据在数据库中原先数据中处于缺失状态,采用其他数据库中L2高频数据转化而来。在读入数据时,期权的成交时间往往很不整齐,采用0.5s读入之间频率读入数据,保证每一条tick读入框架,整理成1min级别分钟交易数据。
与此同时,按照同样的时间格式读入相应的期货数据,由于展期和掉期策略另外构建,所以只读取期货成交量前十名合约数据,而对于期权来说,非主要合约的成交量很小,所以按照昨日成交量进行排序,读取前五名的期权合约数据。
最后得到的结果十对应每一种期货合约品种,有相应的十只期权合约数据,格式统一成日盘和夜盘总计570min分钟数据格式。同时,还建立了期权和期货映射关系,方便计算期权和期货的协同变化因子。


\subsection{第二部分,因子计算}
总计开发了716个基于不同逻辑的因子,主要逻辑有如下类别:

\subsubsection{各类因子类别}
\begin{itemize}
  \item [1)]
  基于BS模型和期权希腊值,各类希腊值组合因子,在期权期货上的映射
  \item [2)]
  传统因子思路:期权CP力量对比,期权多空势力对比,期权期货势力对比,分歧,期权统计因子,各类单纯量价数据乖离,1,2,3,4阶矩,相关性,回归组合因子,基于tick的大小单,
  \item [3)]
  期权定价模型的应用:隐含波动率,隐含波动率组合拆分,基于隐含波动率计算的C,P价格,再套用平价公式和现实价格的对比,波动率对比,计价转换模组一二三
  \item [4)]
  期货期权对冲组合因子,远期期权和近期期权关系因子(拆分欧式和美式期权),收益率和远期利率动态过程
  \item [5)]
  各类早晚盘比例,日夜比例,差值,小时差,30min差,均值,日均值小时均线因子
  \item [6)]
  基于tick的持仓成交量分布因子,Vasicek模型(利用期权定价的概率分布反演收益率曲线),CIR模型
  \item [7)]
  BDT二叉树因子,连续时间对等模型(拆解期权期货和现金的利率,风险,价值做二叉树,利用随机微分方程求瞬时波动率)
  \item [8)]
  其他乱七八糟基于复杂微分方程的积分求导因子,高斯HJM,马尔可夫HJM,BGM,
  \item [9)]
  叠式期权,跨式期权,互换期权,上下限期权等多期权期货组合产品协同因子(在截面上构建相互关系,共同操作,进行套利)
\end{itemize}

\subsubsection{因子筛选和有效因子识别}
因子发掘完毕,利用回测框架进行回测,回测框架主要依赖于我手动开发的因子回测框架,进行了如下几步操作:
\begin{itemize}
  \item [1)]
  残差同方差化,采用统一方差。
  \item [2)]
  因子共线性分析,计算因子线性相关性,大于0.7的剔除,
  \item [3)]
  因子有效性识别,分别采用日月IR读图和NMI指标读图,取并集合。
  \item [4)]
  计算品种间相关性。只在某一两个品种有效的因子剔除。

\end{itemize}
\subsubsection{特征工程}
与此同时,还采取了机器学习方法特征工程方法,采用XGboost,lightGBM和catboost模型,经过1000次迭代,在验证集上认真调参,对因子进行另一种方式的筛选,筛选结果和手动筛选结果不尽相同,分别合成不用的因子。
最后,手工筛选留下了40个超级因子,而机器学习筛选,经过grid search网格调参和贝叶斯优化,得到累计importance很高,留下了100个左右的因子,对这100个因子画PDP图分析,分析其中的非线性形,其中包含了40个R2良好的线性因子,还有其他60个非线性因子,尽管ICIR很低,但是经过逐个数据仔细对比,呈现出良好的非线性价值。


\subsection{第三部分,计算收益率}
计算收益率的过程分为三部分,合成信号,组合截面,和收益计算策略

\subsubsection{合成信号}
合成信号主要有机器学习合成和手工合成两种方法:

\subsubsection{手工合成}
手工合成信号操作主要如下
\begin{itemize}
  \item [1)]
  根据过去一个月,一日,一小时的因子收益率和波动率,对因子筛选,前十名因子进因子池,其中最有效的选池方案是根据月度IR换因子池,每个月换一次
  \item [2)]
  根据因子池内的因子收益和波动率进行加权求和,周期比换因子池周期短,事件证明,根据日度收益率加权效果最好
  \item [3)]
  对合成的信号再次清清洗,去掉不合理的数值

\end{itemize}
\subsubsection{机器学习}
机器学习合成信号的方法,尝试了随机森林,GBDT,简单神经网络训练的方法,调参的时候,将labels设置成不同的return,(包括未来一小时平均return,未来一小时加权平均,未来一小时净值,未来log收益率很多)都进行了训练,其中未来一小时均线效果最好,但是训练还不成熟,信号效果目前和手工合成信号还有差距。然而,手工信号工作量巨大,不仅要读每一各因子的各类图表,还要读每一种模式的各类图表,数据更新以后。再做一遍手工信号实在太累,所以目前的主要精力以机器学习为主,尝试模型组合和模型堆叠,力求达到同样的效果。

\subsubsection{组合截面}
利用信号合成挑出的超级信号和机器学习模型信号进行组合构建,同样采用手工构建和集成学习方法构建两种。
\subsubsection{手工合成}
手工截面构建操作主要如下
\begin{itemize}
  \item [1)]
  对信号进行平滑处理,wavelet滤波操作。
  \item [2)]
  根据品种收益率,品种波动率对55品种截面进行分钟加权,小时加权,除了波动率倒数和收益率以外,主要采用截面收益率协方差矩阵和协方差夏普比率,协方差波动率张量乘积加权,由于每分钟计算矩阵乘积计算量巨大,采用numpy分块矩阵爱因斯坦求和的巧妙方法,让计算时间缩短在5s以内。
  最后经过组合检验,根据小时协方差矩阵和return的乘积分钟加权效果最好
  \item [3)]
  采用不同的换权模式,每小时,每天对数据重新计算
  \item [4)]
  风险控制:根据日风险因子和收益率协方差和残差进行波动率控制和风险控制,对前50\%的波动率和风险权重设置为0。
  \item [5)]
  风险控制:根据上下限设定,单个品种权重不超过阈值,控制风险。
  \item [6)]
  利用二次规划办法求解组合权重分配,考虑上述多个条件,求条件最大值。

\end{itemize}

\subsubsection{机器学习}
采用两层神经网络训练权重,labels设成根据周期变化的未来return,输入是过去的return和信号,经过训练获得了和手工合成相似的答案,但是手工截面经过波动率调整和风险控制,波动率更低,在收益率计算上拥有更低的手续费,所以最好的方法依然是手工截面组合。效果是组合收益是品种简单收益相加的两倍。机器学习方法可能由于调参问题,还需要进一步检验

\subsubsection{收益计算}
机器学习和手工方法用一套收益计算模板,有20多种计算方法,包括日,小时,分钟换仓等,包括
\begin{itemize}
  \item [1)]
  普通分钟换仓,均线分钟换仓,普通小时,均线小时等
  \item [1)]
  分钟换仓,前90\%,80\%信号换仓
  \item [2)]
  分钟换仓,小时换仓,根据截面波动率决定是否触发换仓
  \item [3)]
  对信号分档,1,0,-1或者5档,应用不同周期换仓
  \item [4)]
  利用小波变换操作信号,滤去高阶频率
  \item [5)]
  最后减去基础数据集,用平均值方法设定基础持仓,计算超额收益。

\end{itemize}

其中,超级信号在小时换仓下表现最好可以战胜万4的手续费,部分品种可以战胜万6的手续费,大部分手工信号只能战胜万1的手续费,将机器学习方法截面,能战胜万1的手续费,还需要仔细操作。

\subsection{第四部分,交易结构}
\begin{itemize}
  \item [1)]
  对于收益计算,目前采用的是price的做法,在交易结构上,使用对价方法更新策略净值曲线
  \item [1)]
  考虑换仓时刻的量有多少,根据成交量因子(在因子挖掘中计算得到),设定报价队列,根据量的大小逐步提高价格,当价格超过收益阈值的时候停下,这时得到的额外成本是冲击成本。经过优化对价和报价策略,在期权上目前还是没有办法战胜冲击成本,这是这项研究的一个瓶颈之处。
  \item [2)]
  构建流动性策略,主要参考华泰证券研报,用于降低冲击成本。降低风险
  \item [3)]
  展期策略。展期策略主要是日频策略,是另外一个人构建的,我看过内容,并且在我的交易框架上运行,做了展期日内日夜衔接程序。(主要想法是展期产品计算市场变量,在展期构建远期回购,利用我模拟的利率曲线进行远期回购和曲线拔靴,最后,再用定价模型预估手中的期权价格,择时卖出(变量计算-曲线构建-产品定价))
  \item [4)]
  整套策略被领导拿走,期货公司有返佣,所以实盘结果不得而知。。。

\end{itemize}

\section{第三,深度学习}
\subsection{第一部分,综述}
从10月28日开始,构建高频时间序列数据制作机器学习策略因子,到现在,已经取得了很大的成果,尝试了很多模型,对于最终结构,也有了很大进展,下面是报告内容。
\subsection{第二部分,模型总结}
研究了很多模型,从最简单的CNN,RNN, LSTM, resnet以及各种模型的组合,模型之间各有优劣势,基于目前的研究,利用日频数据,有如下表格(高频数据预测普遍都在0.6以上):
\begin{table}[h!]
\begin{tabular}{lllllll}
\cline{4-4}
model             & mean\_corr & \multicolumn{1}{l|}{sign\_precise} &\multicolumn{1}{l|}{sign\_corr} & 主观评价 & 计算时间 (gpu) & 计算时间(cpu)\\ \cline{4-4}
mobileNet(1d)             & 0.13       & 0.42                               & 0.3                            & 差    & 0.84s & 大约100s\\
mobileNet(2d)             & 0.11       & 0.2                               & 0.44                            & 差    & 1.99s & 大约20min\\
mobileNetV2(1d)            & 0.12       & 不适合                               &                             & 差    & \textless{}0.1s & 15.84s\\
mobileNetV2(2d)             & 0.12       & 不适合                               &                             & 差    & \textless{}0.1s & 20.20s\\
LSTM              & 0.28       & 0.73                               & 0.24                            & 好    & 大约20s & 太慢\\
CNN1d       & 0.14       &                                    &                                 & 差   &  3.55s & 太慢\\
CNN3d       & 0.12       & 0.27                                   &                                 & 差   &  3.26s & 太慢\\
CNN2d       & 0.14       &                                    &                                 & 差   &  8.0s & 太慢\\
CNN1d*2+bottleneck            & 0.08       &                                    &                                 & 差   &  9.3s & 太慢\\
CNN2d*2+bottleneck            & 0.08       &                                    &                                 & 差    & 16.08s & 太慢\\
CNN1d+bottleneck            & 0.14       &  0.48                                  & 0.37                                & 可以   &  & 可以\\
CNN2d+bottleneck            & 0.11       &    0.27                                &    0.26                             & 可以   &  & 可以\\
resCNN+bottleneck-n            & 0.14       &                                    &                                 & 可以   &  & 可以\\
resLSTM+bottleneck-n            & 0.28       & 0.66                                   & 0.22                                & 可以   &  & 可以\\
RNN       & 0.24       &                                    &                                 & 差   &  & 太慢\\
GRU       & 0.26       &                                    &                                 & 差   &  & 太慢\\
resCNN       & 0.13       &                                    &                                 & 差   & 可以\\
resLSTM       & 0.18       &  0.87                                  &  0.68                               & 好   &  大约25s & 很好\\
res-n       & 0.06       &  0.42                                  &                                 & 差   &  较少 & 可以\\
CNN1d-n       & 0.14       &                                    &                                 & 差   &   & 太慢\\
CNN2d-n       & 0.07       &                                    &                                 & 差   &   & 太慢\\
resLSTM-n       & 0.18       &                                    &                                 & 可以   &   & 可以\\
\end{tabular}
\end{table}
\subsection{第三部分,各类研究}
\subsubsection{最终结构}
最终结构参考因诺股票量价因子挖掘结构,有三个人在搭建(包括我),采用多级训练模型,输入是量价数据的分钟,半分钟均线,方差,残差,(高频结构算子挖掘而来),第一级输出是未来分钟,半分钟特征。第二级输入是拟合的第一级输出,第二级输出是未来5分钟,10分钟,复杂结构因子,如MACD,布林带,BM,动量等。本身这些指标已经可以作为很好的因子,经过之前的模型持仓和组合优化尝试,效果已经明显好于手工因子,和经过反复加权的超级因子相比稍差一点,但是这个模型还没有经过finetune,因自此三级模型是用上一级的特征,推理收益率曲线,主要方法还是深度学习。

实际模型一共6级,我们完成了四级,预测周期不断增长,预测指标越来越复杂,类似于预测因子再预测收益率曲线的过程。每一级模型经过resblocks保证每一级都有效,在每一级模型上要进行仔细优化,比如时间,复杂度优化,在此基础上,研究了mobilenet模型,deeplab模型,bottleneck模型,还有复杂门的时间序列循环时间网络(类似LSTM)。神经网络对于高频数据,拥有远超普通线性因子的优越性能,准确率和相关性,互信息率都有很好效果,将来一定能取代普通因子模型。下面对已经研究的东西进行举例说明。(神经网络研究模式大概是这样,一共研究了10种(每周一种))

\subsubsection{bottleneck}
bottleneck方法主要用来训练CNN神经网络,可以降低数据维度,先用bottleneck层作为人为升高数据的隐藏维度,在高维数据中提取信息,因为信息损失,就会减少高频噪声,就可以提取较好的特征,再用bottleneck层人为降低维度,再次进行模型简化,很大的避免了过拟合,算法如下:
\renewcommand{\algorithmicrequire}{\textbf{Input:}}  % Use Input in the format of Algorithm
\renewcommand{\algorithmicensure}{\textbf{Output:}} % Use Output in the format of Algorithm
  \begin{algorithm}[htb]
  \caption{bottleneck structure}
  \label{alg:Framwork}
  \begin{algorithmic}[1]
    \Require
      1 channel 42 * 280000 time series dataloader;
      3*3 线性卷积核;
      5*5 线性卷积核;
    \Ensure
      1 channel 277200 return 或者277200其他高维特征;
    \State 升维convolution(input:1 channel, output:100 channel,kernelsize: 1);
    \label{code:fram:extract}
    \State 维度转换convolution(input:100 channel, output:100 channel,kernelsize: 5);
    \label{code:fram:trainbase}
    \State 特征提取convolution(input:100 channel, output:100 channel,kernelsize: 3);
    \label{code:fram:trainbase}
    \State batchnorm2d;
    \label{code:fram:trainbase}
    \State relu(0.1);
    \label{code:fram:select}
    \State 降维convolution(input:100 channel, output:50 channel,kernelsize: 1);
    \label{code:fram:decrease}
    \State 残差层shortcut;
    \label{code:fram:residual}
    \State 特征提取convolution(input:100 channel, output:50 channel,kernelsize: 3);
    \label{code:fram:residual}
    \State batchnorm2d;
    \label{code:fram:residual}
    \State relu(0.5);
    \label{code:fram:residual}

    \Return residual bottleneck+ residual shortcut;
  \end{algorithmic}
\end{algorithm}

\subsubsection{mobilenet}
mobileNet模型原来用于汽车自动驾驶研究,其效果是能实时的进行训练,运行速度比传统模型大大加快,但是准确率确没有大幅下降,对于多通道数据冗余的数据源有很大意义。对于一般的深度学习模型,都是采用预训练加推理的步骤,首先预训练好各类模型,然后将模型更换配置(如高低频,股市量价,基本面,期权期货CTA)进行模型推理,一日为单位进行训的训练和强化学习(设定环境代理和数据产生他相互作用)。mobilenet的作用是省去模型推理和强化学习步骤,实时训练。

对于股市量价数据而言,对于CNN1d模型,输入的量价类特征,大部分都很类似,具有重复性和数据冗余星,将每一个特征作为输入的channel,很适合使用mobileNet代替CNN模型,或者用mobile blocks代替原始模型中的每一个CNN blocks。大幅降低运算时间,(模型复杂度降低$n^2$,期中n是模型通道数)。操作如下:

\noindent在传统CNN中
$$\fbox{\{\mbox{卷积核的channel}\} = \{\mbox{输入特征矩阵的channel}\} , \{\mbox{输出的新特征矩阵的channel}\} = \{\mbox{卷积核的个数}\}}$$

\noindent在mobilenet中, 将CNN分解成deepwise和pointwise两部分,对于deepwise部分
$$\fbox{\{\mbox{卷积核的个数}\} = \{\mbox{输入特征矩阵的channel}\} = \{\mbox{输出的新特征矩阵的channel}\}}$$
$$\fbox{\{\mbox{卷积核的个channel} \} = 1}$$

\noindent对于pointwise部分
$$\fbox{\{\mbox{1*1卷积核的个数}\} = \{\mbox{输出特征矩阵的channel}\} , \{\mbox{输出的新特征矩阵的channel}\} = \{\mbox{1*1卷积核的个数}\}}$$

\renewcommand{\algorithmicrequire}{\textbf{Input:}}  % Use Input in the format of Algorithm
\renewcommand{\algorithmicensure}{\textbf{Output:}} % Use Output in the format of Algorithm
  \begin{algorithm}[htb]
  \caption{mobile structure}
  \label{alg:Framwork}
  \begin{algorithmic}[1]
    \Require
      1 channel 42 * 280000 time series dataloader;
      3*3 线性卷积核;
      5*5 线性卷积核;
    \Ensure
      1 channel 277200 return 或者277200其他高维特征;
    \State 42 个  1 channel CNN(input:1 channel, output:42, channel,kernelsize: 3);
    \label{code:fram:extract}
    \State batchnorm2d(1 feature);
    \label{code:fram:trainbase}
    \State relu;
    \label{code:fram:trainbase}
    \State residual net;
    \label{code:fram:trainbase}
    \State relu(0.1);
    \label{code:fram:select}
    \State 1 个 42 channel CNN(input:42 channel, output:1, channel,kernelsize: 1);
    \label{code:fram:decrease}
    \State batchnorm2d(42 features);
    \label{code:fram:residual}
    \State relu(0.5);
    \label{code:fram:residual}
    \State residual shortcut;

    \label{code:fram:residual}
    \Return residual net+ residual shortcut;
  \end{algorithmic}
\end{algorithm}
前面说的mobilenet的基准模型,但是有时候你需要更小的模型,就是mobilenetV2模型。这里引入了两个超参数:width multiplier和resolution multiplier。第一个参数width multiplier主要是按比例减少通道数,
第二个参数resolution multiplier主要是按比例降低数据长度。

降低后的计算精度肯定会下降,但是进一步提高了计算效率,对于时间序列这种超大hidden size的模型来说,非常可取。因此加入了这两个参数进行尝试。

现在自动驾驶策略和手机算法推荐普遍采用的是mobilenetV3模型,是mobilenetV2的进一步变化,首先,使用bottleneck代替CNN参数,第二,去掉了拖慢时间的relu,第三,纺锤形
bottleneck和mobilenet本质上存在一层的计算重复,使用一层1*1 CNN就可以实现。

总网络不算dropout和残差网络,一共15层,使用mobileNet进行训练,训练速度比CNN提高了80\%左右,训练精度几乎没有下降。我觉得完全可以实现,在训练精度保持不变的情况下,计算时间大幅下降,训练精度大大提高,性能超过bottleneck+CNN的训练效果。



\subsection{第四,神经科学方法研究}
attention,神经科学的主要方法,目前最主流的是基于transformer体系encoder,decoder模型,围绕transformer时间序列研究展开,已经做了很多尝试。

基于encoder直接进行预测,效果比较差,将中间向量直接连接全连接神经网络,训练相关性大约在8\%,如果在encoder的attention层和forward层加入residuel机制进行尝试,在中间向量C中每一层都加入dropout。

在中间向量和前馈神经网络中加入RNN,LSTM,作为decoder,发现效果比原来好。那么合理的想法就是在transformer的框架内,在中间向量,前馈神经网络中加入新的attention系统(即multi-head attention机制),再用残差神经网络连接,最后做layer normalize,这样做有数据结构性困难,效果也不太好,还需要进一步调整神经结构。另外,transformer结构在高频数据上不尽如人意,但是在股票基本面数据上,效果很好,可能和股票基本面数据的金融逻辑并非number逻辑而是金融文本逻辑有关。

\subsection{第五,高性能计算}
对GPU底层构架有一定了解,深度学习离不开高性能计算。大规模的深度学习依赖超算,集群计算机,能申请到大量的计算资源是关键,第二还要对这方面很了解,比如slurm,跳板机,多卡训练,分布式计算(dusk),multiprocessing,这些我平时都有使用,GPU训练还要注重服务器通信,宽带系统,内存泄露,这些知识是必要的,目前我每个部分都有一定尝试,还需要系统的学习。高性能计算非常重要,一个高性能计算水平高的人是深度学习团队的基石。

\section{第四,其他学习}
\subsection{阅读论文}
阅读研报华泰人工智能系列1-54。
微软qlib论文

\begin{thebibliography}{1}
\bibitem{ref1} K.Jensen, V.M.Acosta, J.M.Higbie, et al. Cancellation of nonlinear Zeeman shifts with light shifts[J]. Physical Review A, 2009, 79(2):023406.
\bibitem{ref1} 华泰人工智能系列
\end{thebibliography}

\end{document} 